\chapter{Time and Causality}\label{chapter:timeandcausality}

\paragraph{}This undertaking will not discuss the current understandings of time and events from a scientific perspective; the work of Roger Penrose in Cycles of Time\footcite[]{penrose2011cycles} would suffice as a basic introduction to the scientific concepts, whilst for those seeking a more rigorous understanding, I would recommend the works of Albert Einstein on his theories of relativity\footcite[]{einstein2001relativity}, both General and Special cases, and the studies on thermodynamics, entropy and the arrow of time by Stephen Hawking\footcite[]{hawking1988a}.

\paragraph{}We shall begin by looking at the intuitive and implicit understanding, and thus the logical reasoning and conclusions, we as humans draw on the concepts of time, events, connectedness of events and causality. Consequence of action is the first form of causal reasoning we are taught, in the form of parental disciplining, but I would argue that it is inherent, for a child that was not “disciplined” would still reason causally of action to consequence.

\paragraph{}If I knock this table, the vase atop it will topple, fall, and break, and thus we causally link and reason the knocking with the consequential action of the vase breaking. Although we do possess a rigorous mechanical understanding of such events, such as the table supports the vase, if a force is exerted upon the table, and the table is moved, this force translates to the vase, the vase preceding in a stable equilibrium position proceeds to tilt due to this jolt, if the force was large enough this moves the centre of mass and its subsequent line of action out of its supportive base, and the object, the vase, then seeks a new state of equilibrium causing it to topple, and consequentially break when hitting the surface of the floor.

\paragraph{}It must be noted that regardless of the rigorous mathematical and mechanical models, it can be said that all humans possess the ability to reason thus, regardless of the knowledge of these models, if they are only willing to do so. It can also further be argued that the mechanical and mathematical models are born from that reasoning, and the desire to understand our world with objectivity. And although any human is capable of reasoning in the same manner, for the notion of causality, or otherwise, it can hence be shown that the limit of reasoning is attributed only by as much as the lack of knowledge that said being possesses\footnote{Postulate 2 as defined in \autoref{chapter:axiomsofreason}}. Therefore, from reasoning is born knowledge, and with knowledge our reasoning becomes deeper and more intricate; reasoning is thus self-perpetuating in its understanding and growth.