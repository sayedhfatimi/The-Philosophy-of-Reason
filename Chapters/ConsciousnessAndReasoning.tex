\chapter{Consciousness and Reasoning}\label{chapter:consciousnessandreasoning}

\paragraph{}Descartes in his meditations\footcite{descartes1986meditations} is perhaps the first, as far as I am aware, to approach consciousness from an axiomatic basis, his philosophy formalises on the basic statement of \emph{``Cogito, Ergo Sum''\footnote{Loosely translated: ``I think, therefore I am''}} implying the human conscious, in the least if not the being, is merely a collection of thoughts, but what are thoughts if not what we are arguing for in this publication as the reasonings and logic of our cognitive functions.

\paragraph{}Whilst many would argue his philosophy is flawed\footcite[A great summary of some retorts can be found in Chapter 1 of][]{gottlieb2016the}, that in later chapters he proceeds to make assumptions, rather than going through the rigor of developing that foundation for all concepts he proposes, as he did for the philosophy to begin with, it is nevertheless well established that our understanding of consciousness and reasoning, as well as that of modern philosophy has been bought about from the development of his initial work. 