\chapter{Axioms of Reason}\label{chapter:axiomsofreason}

\paragraph{}In the endeavour to discover a philosophy of conscious reasoning, it is essential to define the axioms of reason, invariably looking at consciousness itself, so whilst in this publication we may study the philosophy of reason, it is inevitable that we delve into a study of the psychology of consciousness to bring about a formulation of our final theory, albeit to the most minimal extent.

\paragraph{}The psychological study of consciousness has been examined at length by men of better understanding of it than myself, and I would refer my readers to the studies of those such as Carl Gustav Jung\footcite{jung2017modern} or Friedrich Wilhelm Nietzsche\footcite{nietzsche2014beyond}, without whom this study would most likely have never existed.

\paragraph{}It is ironic that for the study of reasoning we must first develop axiomatic foundations, for what are axioms other than a well-established basis of reasoning from which reason follows; that is to say we will reason that all reason comes from that basis of the following postulates of reason, quite a mouthful.

\paragraph{}Here we will propose each postulate and formalize a study of them later:

\begin{postulate}
    Reasoning is invariant of language
\end{postulate}

\begin{postulate}
    The limits of reason arise from a lack of reasoning
\end{postulate}

\begin{postulate}
    Paradoxes of reason are reasonable
\end{postulate}

\begin{postulate}
    Reasons are not global truths
\end{postulate}

\paragraph{}The axioms of reason are proposed without hitherto any necessity of explanation and will be accepted as they are, to formulate our theory.