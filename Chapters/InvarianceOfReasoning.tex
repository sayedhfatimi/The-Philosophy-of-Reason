\chapter{Invariance of Reasoning}\label{chapter:invarianceofreasoning}

\paragraph{}In our theory, it is formally accepted as the first postulate that reasoning is invariant, but this brings with it the question, ``What is meant when we say that reasoning is invariant of language?''

\paragraph{}In the most simple form we argue that reasoning of all kind, whether it be cognitive, verbal, illustrative, written or otherwise, is fundamentally the same, and that they rather take different forms, the subjective perspective of which can be different from one individual to the next, but the underlying mechanisms are all the same.

\paragraph{}We shall begin with asking ourselves a simple question, ``Where are we?'', my senses would reason that I am in a room, my cognition would reason I am at home, an illustrative map showing my current geolocation via GPS would show my precise address relative to the frame of reference in which its developed, diving deeper I could reason that I am on earth, in the solar system, in the milky way galaxy, within the Orion-Cygnus Arm, in the universe and so forth and so forth.

\paragraph{}The language in which all these forms of reasoning exhibit themselves may vary but in essence they portray the same information, whether precisely or ambiguously, does not matter; we leave that study for the scientist and mathematician to figure out, what’s important is regardless of the form of language chosen, the reasoning is the same.

\paragraph{}This is a rather difficult supposition to accept, as it implies that whatever I am saying in answering the question, I am necessarily saying the same thing, but how can that possibly be, how can a precise latitude and longitude be equivalent to the vague response of ``home'', and the answer is, geographically, mathematically they may not be, but from the perspective of a topological understanding of language it is.

\paragraph{}Take for example a map of Paris Metro or London Underground, when it is asked for how we get from point A to point B, assuming A and B exist as locations on our map, we can necessarily formulate a route of travel getting us from our origin through multiple intersections to our destination, all appearing on a straight line, yet if we plot the same route on a top down map or atlas they would appear to be completely different routes, twisting and turning, not at all straight.

\paragraph{}It is of course known that the first map is what is mathematically called a topological map, where the presented distances do not represent true distances from the real world, but the two different forms of information, or reasoning, presented of routes from origin to destination, necessarily portray the same. This is what is meant by the invariance of reasoning, and what is postulated by our first axiom, the maps ``speak'' different languages, but reason the same information.
